%
% COLOR CODE
% Kai: Red
%
\documentclass[12pt,a4paper]{amsart}
\usepackage{amsthm,amsfonts,amsmath,amssymb,latexsym}
\usepackage{epsfig,graphics,color}
%\usepackage{graphicx}
%\usepackage{epsfig}
%\usepackage{graphicx,color}
\usepackage[all]{xy}
\usepackage[breaklinks=true]{hyperref}
%\usepackage{showkeys}
\usepackage{mathrsfs}
%\usepackage{makeidx}
%\usepackage{cancel}
\usepackage{stmaryrd}
\usepackage{verbatim}
\usepackage{bm}
\usepackage{mathabx}
%\usepackage[showframe=true]{geometry}
\usepackage{enumitem}
\usepackage{pgf,amsmath,tikz,type1cm,fix-cm}
\usetikzlibrary{positioning}
\usepackage{hyperref}

\newlength{\XHeight}
\newlength{\XWidth}
%
\setlength{\parskip}{5pt} 
\setlist[itemize,1]{leftmargin=\dimexpr 26pt-.1in}

%%%%%%%%%%%%%%%%  header %%%%%%%%%%%%%%%%%



%
%%%%%%%%%%%%%%%% Environments/macros %%%%%%%%%%
%
%default is \theoremstyle{theorem}. 
%But there are also other styles !!
\newtheorem{PARA}{}[section]
\newtheorem{theorem}[PARA]{Theorem}
\newtheorem{thm}[PARA]{Theorem}
\newtheorem{corollary}[PARA]{Corollary}
\newtheorem{cor}[PARA]{Corollary}
\newtheorem{lemma}[PARA]{Lemma}
\newtheorem{proposition}[PARA]{Proposition}
\newtheorem{prop}[PARA]{Proposition}
\newtheorem{definition}[PARA]{Definition}
\newtheorem{defi}[PARA]{Definition}
\newtheorem{definition-proposition}[PARA]{Definition-Proposition}
\newtheorem{definition-lemma}[PARA]{Definition-Lemma}
\newtheorem{conjecture}[PARA]{Conjecture}
\newtheorem{question}[PARA]{Question}
\theoremstyle{definition}
\newtheorem{remark}[PARA]{Remark}
\newtheorem{rem}[PARA]{Remark}
\theoremstyle{theorem}
\newtheorem{example}[PARA]{Example}
%
\newcommand{\para}{\begin{PARA}\rm}
\newcommand{\arap}{\end{PARA}\rm}
\newcommand{\dfn}{\begin{definition}\rm}
\newcommand{\nfd}{\end{definition}\rm}
\newcommand{\rmk}{\begin{remark}\rm}
\newcommand{\kmr}{\end{remark}\rm}
\newcommand{\xmpl}{\begin{example}\rm}
\newcommand{\lpmx}{\end{example}\rm}
%
\newcommand{\jdef}[1]{{\bf #1}}
\newcommand{\ds}{\displaystyle}
\newcommand{\Iso}{\mathrm{Iso}}
\newcommand{\Times}[2]{{}_{#1}\!\!\times_{#2}}
\newcommand{\liftbox}{\par\vspace{-2em}}
%
\newcommand{\cA}{\mathcal{A}}
\newcommand{\cB}{\mathcal{B}}
\newcommand{\cBbar}{\overline{\mathcal{B}}}
\newcommand{\cC}{\mathcal{C}}
\newcommand{\cD}{\mathcal{D}}
\newcommand{\cF}{\mathcal{F}}
\newcommand{\cE}{\mathcal{E}}
\newcommand{\cG}{\mathcal{G}}
\newcommand{\cH}{\mathcal{H}}
\newcommand{\cI}{\mathcal{I}}
\newcommand{\cJ}{\mathcal{J}}
\newcommand{\cK}{\mathcal{K}}
\newcommand{\cL}{\mathcal{L}}
\newcommand{\cM}{\mathcal{M}}
\newcommand{\cMbar}{\overline{\mathcal{M}}}
\newcommand{\cN}{\mathcal{N}}
\newcommand{\cO}{\mathcal{O}}
\newcommand\co{
  \mathchoice
    {{\scriptstyle\mathcal{O}}}% \displaystyle
    {{\scriptstyle\mathcal{O}}}% \textstyle
    {{\scriptscriptstyle\mathcal{O}}}% \scriptstyle
    {\scalebox{.7}{$\scriptscriptstyle\mathcal{O}$}}%\scriptscriptstyle
  }
\newcommand{\cP}{\mathcal{P}}
\newcommand{\cQ}{\mathcal{Q}}
\newcommand{\cR}{\mathcal{R}}
\newcommand{\cS}{\mathcal{S}}
\newcommand{\cT}{\mathcal{T}}
\newcommand{\cU}{\mathcal{U}}
\newcommand{\cUbar}{\overline{\mathcal{U}}}
\newcommand{\cV}{\mathcal{V}}
\newcommand{\cW}{\mathcal{W}}
\newcommand{\cX}{\mathcal{X}}
\newcommand{\cY}{\mathcal{Y}}
\newcommand{\cZ}{\mathcal{Z}}
%
\newcommand{\oeps}{{\overline{\epsilon}}}
\newcommand{\og}{{\overline{\gamma}}}
\newcommand{\ug}{{\underline{\gamma}}}
\newcommand{\oev}{\overline{\mathrm{ev}}}
\newcommand{\uev}{\underline{\mathrm{ev}}}
\newcommand{\oS}{{\overline{S}}}
\newcommand{\uS}{{\underline{S}}}
\newcommand{\oU}{\overline{U}}
\newcommand{\uU}{\underline{U}}
\newcommand{\oV}{\overline{V}}
\newcommand{\uV}{\underline{V}}
\newcommand{\opsi}{\overline{\Psi}}
\newcommand{\upsi}{\underline{\Psi}}
\newcommand{\ozeta}{\overline{\zeta}}
\newcommand{\uzeta}{\underline{\zeta}}
\newcommand{\otheta}{\overline{\theta}}
\newcommand{\utheta}{\underline{\theta}}
%\DeclareMathOperator{\hatotimes}{{\hat{\otimes}}}
\newcommand{\hatotimes}{\, {\wh\otimes}}
%
\newcommand{\Rectangle}[8]{
\begin{array}{ccc}
{#1} & \mapright{#2} & {#3}\\
\mapdown{#4} && \mapdown{#5} \\
{#6} & \mapright{#7} & {#8}
\end{array}
}
\newcommand{\mapdown}[1]{\Big\downarrow
\rlap{$\vcenter{\hbox{$\scriptstyle#1$}}$}}
\newcommand{\mapright}[1]{\smash{
\mathop{\longrightarrow}\limits^{#1}}}
%
\newcommand{\medskipnoindent}{\par\medskip\par\noindent}
%
%
\newcommand{\one}
{{{\mathchoice \mathrm{ 1\mskip-4mu l} \mathrm{ 1\mskip-4mu l}
\mathrm{ 1\mskip-4.5mu l} \mathrm{ 1\mskip-5mu l}}}}
%
\newcommand{\dslash}{/\mskip-6mu/}
%
%
\newcommand{\A}{{\mathbb{A}}}
\newcommand{\B}{{\mathbb{B}}}
\newcommand{\C}{{\mathbb{C}}}
\newcommand{\D}{{\mathbb{D}}}
\newcommand{\F}{{\mathbb{F}}}
\renewcommand{\H}{{\mathbb{H}}}
\newcommand{\LL}{{\mathbb{L}}}
\newcommand{\M}{{\mathbb{M}}}
\newcommand{\N}{{\mathbb{N}}}
\newcommand{\Q}{{\mathbb{Q}}}
\newcommand{\R}{{\mathbb{R}}}
\renewcommand{\SS}{{\mathbb{S}}}
\newcommand{\T}{{\mathbb{T}}} 
%\renewcommand{\u}{{\mathbf{u}}}
%\renewcommand{\v}{{\mathbf{v}}}
\newcommand{\W}{{\mathbb{W}}}
\newcommand{\x}{{\mathbf{x}}}
\newcommand{\Z}{{\mathbb{Z}}}
%
%\newcommand{\coker}{\mathrm{ coker }}  % cokernel
\DeclareMathOperator{\coker}{coker}
%\newcommand{\colim}{\mathrm{ colim}\, }  % colimit
\DeclareMathOperator*{\colim}{colim}
\newcommand{\Cr}{\mathrm{ Cr}\, }  % Critical value
\newcommand{\im}{\mathrm{im}\,}        % image
\newcommand{\range}{\mathrm{ range }}  % range
\newcommand{\SPAN}{\mathrm{ span }}    % span
\newcommand{\Det}{{\mathrm{Det}}}      % Det
\newcommand{\dom}{\mathrm{ dom }}      % domain
\newcommand{\DIV}{\mathrm{ div }}      % divergence
\newcommand{\trace}{\mathrm{ trace }}  % trace
\newcommand{\tr}{\mathrm{ tr }}        % trace
\newcommand{\sign}{\mathrm{ sign }}    % sign
\newcommand{\Gr}{\mathrm{ Gr}\, }  % Graded object
\newcommand{\id}{\mathrm{ id}}         % identity
\newcommand{\Id}{\mathrm{ Id}}
\newcommand{\rank}{\mathrm{ rank }}    % rank
\newcommand{\codim}{\mathrm{ codim}}   % codimension
\newcommand{\diag}{\mathrm{ diag}}     % diagonal matrix
\newcommand{\cl}{\mathrm{ cl}}         % closure
\newcommand{\dist}{\mathrm{ dist}}     % distance
\newcommand{\INT}{\mathrm{ int}}       % interior
\newcommand{\supp}{\mathrm{ supp}}     % support
\newcommand{\INDEX}{\mathrm{ index}}   % (Fredholm)index
\newcommand{\ind}{\mathrm{ind}}
\newcommand{\grad}{\mathrm{ grad }}    % gradient
\renewcommand{\Re}{\mathrm{ Re\,}}       % real part
\renewcommand{\Im}{\mathrm{ Im\,}}       % imaginary part
\newcommand{\Met}{{\mathfrak{Met}}}       % space of metrics
\newcommand{\Jreg}{\cJ_{\mathrm{reg}}}   %regular J's
\newcommand{\Jphireg}{\cJ_{\phi,\mathrm{reg}}}
\newcommand{\Freg}{\cF_{\mathrm{reg}}}
\newcommand{\pr}{{\mathrm{pr}}}
\newcommand{\ev}{\mathrm{ev}}
\newcommand{\Ev}{\mathrm{Ev}}
\newcommand{\Aut}{\mathrm{ Aut}}          % Automorphisms
\newcommand{\Diff}{\mathrm{ Diff}}        % Diffeomorphisms
\newcommand{\Vect}{\mathrm{ Vect}}        % Vector fields
\newcommand{\Hol}{\mathrm{ Hol}}          % Holomorphic
\newcommand{\End}{\mathrm{ End}}          % Endomorphisms
\newcommand{\G}{\mathrm{G}}
\newcommand{\Lie}{\mathrm{Lie}}
\newcommand{\PSL}{\mathrm{PSL}}
\newcommand{\Op}{{O}{p}}
\newcommand{\Int}{\mathrm{Int}}
\newcommand{\Hom}{\mathrm{Hom}}
\newcommand{\Fix}{\mathrm{Fix}}
\newcommand{\fin}{\mathrm{fin}}
\newcommand{\Top}{\mathrm{Top}}
%
%\renewcommand{\phi}{{\varphi}}
\newcommand{\eps}{{\varepsilon}}
\newcommand{\om}{{\omega}}
\newcommand{\Om}{{\Omega}}
\newcommand{\TOm}{{\widetilde{\Omega}}} 
\newcommand{\tc}{{\widetilde{c}}}
\newcommand{\tJ}{{\widetilde{J}}}
\newcommand{\tcO}{{\widetilde{\cO}}}
%\newcommand{\ttcO}{{\widetilde{\!\widetilde{\cO}}}}
\newcommand{\ttcO}{{\stackrel{\raisebox{-1mm}{$\approx$}}{\raisebox{.01mm}{$\cO$}}}}
\newcommand{\ttcOs}{{\stackrel{\raisebox{-1mm}{$\approx$}}{\raisebox{.01mm}{$\cO$}}{}\hspace{-1mm}^s}}
\newcommand{\ttcOu}{{\stackrel{\raisebox{-1mm}{$\approx$}}{\raisebox{.01mm}{$\cO$}}{}\hspace{-1mm}^u}}
\newcommand{\tS}{{\widetilde{S}}}
\newcommand{\tcS}{{\widetilde{\cS}}}
\newcommand{\tcSg}{{\widetilde{\cS}_{\textrm{good}}}}
\newcommand{\tcSb}{{\widetilde{\cS}_{\textrm{bad}}}}
\newcommand{\tw}{{\widetilde{w}}}
\newcommand{\tx}{{\widetilde{x}}} 
\newcommand{\ty}{{\widetilde{y}}}
%
\newcommand{\Cinf}{C^{\infty}}
\newcommand{\reg}{\mathrm{ reg}}
\newcommand{\CZ}{\mathrm{CZ}}
%
\newcommand{\inner}[2]{\left\langle #1, #2\right\rangle}
\newcommand{\winner}[2]{\left\langle #1{\wedge}#2\right\rangle}
%
\def\NABLA#1{{\mathop{\nabla\kern-.5ex\lower1ex\hbox{$#1$}}}}
\def\Nabla#1{\nabla\kern-.5ex{}_{#1}}
\def\Tabla#1{\Tilde\nabla\kern-.5ex{}_{#1}}
%
\def\abs#1{\mathopen|#1\mathclose|}
\def\Abs#1{\left|#1\right|}
\def\norm#1{\mathopen\|#1\mathclose\|}
\def\Norm#1{\left\|#1\right\|}
\def\dslash{/\mskip-6mu/}
%
\renewcommand{\Tilde}{\widetilde}
\renewcommand{\Hat}{\widehat}
\newcommand{\half}{\mbox{$\frac12$}}
\newcommand{\p}{{\partial}}
\newcommand{\dbar}{{\bar\partial}}
\newcommand{\TopVect}{\mathrm{TopVect}}


\newenvironment{enum}
{\begin{enumerate}\renewcommand{\labelenumi}{(\roman{enumi})}}
{\renewcommand{\labelenumi}{(\arabic{enumi}.}\end{enumerate}}

% Kai's definitions %%%
\newcommand{\la}{\langle}
\newcommand{\ra}{\rangle}
\newcommand{\wh}{\widehat}
\newcommand{\ol}{\overline}
\newcommand{\ul}{\underline}
\newcommand{\MM}{\mathcal{M}}
\newcommand{\NN}{\mathcal{N}}
\newcommand{\Ad}{{\rm Ad}}
%\newcommand{\Diff}{{\rm Diff}}
\newcommand{\ContS}{{\rm ContS}}
\newcommand{\ContF}{{\rm ContF}}
\newcommand{\Crit}{{\rm Crit}}
\newcommand{\wt}{\widetilde}
\newcommand{\fk}{\frak{k}}
\newcommand{\TQFT}{\rm TQFT}
\newcommand{\wc}{\widecheck}
\newcommand{\K}{\mathbb{K}}
\newcommand{\into}{\hookrightarrow}
\newcommand{\pb}{\overline{\partial}}
\newcommand{\bk}{\mathbf{k}}
%

\newcommand{\umu}{{\underline{\mu}}}
\newcommand{\um}{{\underline{m}}}
\newcommand{\utau}{{\underline{\tau}}}
\newcommand{\usigma}{{\underline{\sigma}}}
\newcommand{\ubeta}{{\underline{\beta}}}
\newcommand{\uc}{{\underline{c}}}
\renewcommand{\uzeta}{{\underline{\zeta}}}

%

%\definecolor{darkgreen}{rgb}{0.0, 0.2, 0.13}
\definecolor{darkgreen}{rgb}{0.12, 0.3, 0.17}
\definecolor{burntorange}{rgb}{0.8, 0.33, 0.0}
\definecolor{chromeyellow}{rgb}{1.0, 0.65, 0.0}
\definecolor{darkorange}{rgb}{1.0, 0.55, 0.0}
\definecolor{flame}{rgb}{0.89, 0.35, 0.13}
\definecolor{lightgray}{rgb}{0.75, 0.75, 0.75}
%
\parskip=4pt
\parindent=0pt

%circled symbols
%
%Example: Hello, \mymk{1,2,3,4,5} world \circled{1}
%
\usetikzlibrary{arrows}
\usetikzlibrary{shapes}
\newcommand{\mymk}[1]{%
  \tikz[baseline=(char.base)]\node[anchor=south west, draw,rectangle, rounded corners, inner sep=2pt, minimum size=7mm,
    text height=2mm](char){\ensuremath{#1}} ;}

\newcommand*\circled[1]{\tikz[baseline=(char.base)]{
            \node[shape=circle,draw,inner sep=2pt] (char) {#1};}}

\newcommand{\boldbeta}{{\boldsymbol{\beta}}}
\newcommand{\boldlambda}{{\boldsymbol{\lambda}}}
\newcommand{\boldmu}{{\boldsymbol{\mu}}}
\newcommand{\boldeta}{{\boldsymbol{\eta}}}
\newcommand{\boldeps}{{\boldsymbol{\eps}}}
\newcommand{\boldc}{{\boldsymbol{c}}}
\newcommand{\boldp}{{\boldsymbol{p}}}
\newcommand{\boldB}{{\boldsymbol{B}}}
\newcommand{\boldS}{{\boldsymbol{S}}}
\newcommand{\boldzeta}{{\boldsymbol{\zeta}}}
%
\newcommand{\Spec}{{\rm Spec}}
\newcommand{\Cont}{{\rm Cont}}
\newcommand{\Map}{{\rm Map}}
\newcommand{\disc}{{\rm discr}}


\setcounter{tocdepth}{1}

% Kai's color is Red
\newcommand{\kai}{\color{red}}
%\usepackage[dvipsnames]{xcolor}

\begin{document}

\title{Notes on Condensed Mathematics} 
\author{Kai Cieliebak, Milan Zerbin, ...}
%\address{Universit\"at Augsburg \newline Universit\"atsstrasse 14, D-86159 Augsburg, Germany}
%\email{kai.cieliebak@math.uni-augsburg.de}
\date{\today}


%%%%%%%%%%%%%%%%%%%%%%%%%%%%%%%%%%%%%%%%%%%%%%
%%%%%%%%%%%%%%%%%%%%%%%%%%%%%%%%%%%%%%%%%%%%%%
%%%%%%%%%%%%%%%% Abstract %%%%%%%%%%%%%%%%%%%%
%%%%%%%%%%%%%%%%%%%%%%%%%%%%%%%%%%%%%%%%%%%%%%
%%%%%%%%%%%%%%%%%%%%%%%%%%%%%%%%%%%%%%%%%%%%%%

%\begin{abstract} 
%\end{abstract}

\maketitle

\tableofcontents


%%%%%%%%%%%%%%%%%%%%%%%%%%%%%%%%%%%%%%%%%%%%%%%%%
\section{Introduction}\label{sec:intro}
%%%%%%%%%%%%%%%%%%%%%%%%%%%%%%%%%%%%%%%%%%%%%%%%%

%%%%%%%%%%%%%%%%%%%%%%%%%%%%%%%%%%%%%%%%%%%%%%%%%
\section{Light condensed sets}\label{sec:light-condensed-sets}
%%%%%%%%%%%%%%%%%%%%%%%%%%%%%%%%%%%%%%%%%%%%%%%%%

The ordinary text in this section is a transcript of Scholze's Lecture
2 in~\cite{Clausen-Scholze-IHES}, whose video is available under\\
https://www.youtube.com/playlist?list=PLx5f8IelFRgGmu6gmL-Kf$\_$Rl$\_$6Mm7juZO

Text in colours such as {\kai this one} contains details and
explanations from our side. 

Other relevant references are the lecture
notes~\cite{Scholze-condensed,Scholze-analytic,Clausen-Scholze-complex}. 

%%%
\subsection{Light profinite sets}
%%%

Consider the following three categories.

1) {\em Profinite sets} {\bf ProFin} \\
Objects: $\lim_{i\in I}S_i$ for $S_i$ finite sets, $I$ cofiltered poset
($\forall i,j \exists k\leq i,j$)
\\
Morphisms: 
$$
  \Hom(\lim_{i\in I}S_i,\lim_{j\in J}T_j) := \lim_j\colim_i\Map(S_i,T_j)
$$

2) {\em Totally disconnected compact Hausdorff spaces} {\bf DiscComp}

3) {\em Boolean algebras} {\bf Boolean} \\ 
Objects: commutative rings $A$ such that $x^2=x$ for all $x$ 

\begin{prop}[Stone Duality]
There are equivalences of categories
$$
  \textbf{ProFin} \stackrel{\sim}\longrightarrow 
  \textbf{DiscComp} \stackrel{\sim}\longrightarrow  
  \textbf{Boolean}^{\rm op}.
$$
\end{prop}

\begin{proof}
For 1) $\to$ 2) $\to$ 3), equip $S=\lim S_i$ with the (inverse,
projective) limit topology and assign to it the Boolean algebra
$$
   \Cont(S,\F_2) = \colim_i\Map(S_i,\F_2).
$$
Conversely, assign to $A$ the profinite set
$$
  \Spec A = \Hom(A,\F_2) = \lim_{A_i\subset A\text{ finite}}\Hom(A_i,\F_2).
$$
\end{proof}

There are two measures how ``big'' a profinite set is.

\begin{defi}
Let $S$ be a profinite set. 
\begin{itemize}
\item The {\em size} of $S$ is $\kappa=|S|$.
\item The {\em weight} of $S$ is $\lambda=|\Cont(S,\F_2)|$.
\item $S$ is {\em light} if $\lambda\leq\om=|\N|$.
\end{itemize}
\end{defi}

\begin{rem}
If $\lambda$ is infinite, then $\lambda$ is the smallest
  cardinality of an index set $I$ such that $S=\lim_{i\in I}S_i$.
Thus, $S$ is light iff is is a countable limit of finite sets.
\end{rem}

\begin{example}[Examples of profinite sets]
(0) Finite sets.
  
(1) $S = \N\cup\{\infty\} = \lim_n\{0,1,\dots,n,\infty\}$ has $\kappa=\lambda=\om$. 

(2) The {\em Cantor set} $S = \{0,1\}^\N = \lim_n\{0,1\}^n$ has
  $\kappa=2^\om$ and $\lambda=\om$.

(3) The {\em Stone-\v{C}ech compactification} $S=\beta\N$ satisfies 
$$
  \Cont(\beta\N,\F_2) = \Cont(\N,\F_2) \cong \{\text{subset of }\N\},
$$
so it has $\kappa=2^{2^\om}$ and $\lambda=2^\om$. 
\end{example}

\begin{prop}
(a) We always have $\lambda\leq 2^\kappa$ and $\kappa\leq 2^\lambda$. \\
(b) If $\kappa=\om$ then $\lambda=\om$. 
\end{prop}

\begin{proof}
(a) follows from
\begin{align*}
  \lambda &= |\Cont(S,\F_2)| \leq |\Map(S,\F_2)| = 2^\kappa, \cr
  \kappa &= |\Hom(A,\F_2)| \leq |\Map(A,\F_2)| = 2^\lambda.
\end{align*}
(b) Write $S=\{s_0,s_1,s_2,\dots\}$ and choose quotients
$S\twoheadrightarrow S_n$ which restrict to injections
$\{s_0,\dots,s_n\}\into S_n$. Then the induced map $S\to\lim_n S_n$ is
injective and surjective.
%and thus $\Cont(S,\F_2)=\colim_n\Cont(S_n,\F_2)$ is countable.   
\end{proof}

\begin{rem}
The same proof shows: if $\kappa$ is infinite then $\lambda\leq\kappa$.
\end{rem}

\begin{prop}
The following categories are equivalent to the category of light
profinite sets:

1) $\text{\bf Pro}_{\N}\text{\bf Fin}$ \\
Objects: $\lim_{n\in\N}S_n$ for $S_n$ finite sets
\\
Morphisms: 
\begin{align*}
  \Hom(\lim_{n\in\N}S_n,\lim_{m\in\N}T_m)
  &:= \lim_m\colim_n\Map(S_n,T_m) \cr
  &= \colim_f\lim_m\Map(S_{f(m)},T_m)
\end{align*}
over strictly increasing maps $f:\N\to\N$.
  
2) {\em Metrizable} totally disconnected compact Hausdorff spaces

3) $(\text{{\em Countable} Boolean algebras})^{\rm op}$
\end{prop}

\begin{prop}
The category of light profinite sets has all countable
limits. Sequential limits of surjections are surjective. 
\end{prop}

\begin{prop}
For each light profinite set $S$ there exists a surjection
$\{0,1\}^\N\to S$. 
\end{prop}

\begin{proof}
Exercise (simple induction).
\end{proof}

The following two propositions are special to {\em light} profinite
sets. 

\begin{prop}
Let $S$ be a {\em light} profinite set. Then every open subset $U\subset S$
is a countable union of light profinite sets. 
\end{prop}

\begin{proof}
Write $S=\lim_{n\in\N}S_n$ with $S_n$ finite and the canonical maps
$\pi_n:S\to S_n$. Denote $Z:=S\setminus U$ and $Z_n:=\pi_n(Z)\subset S_n$.
Then $U=\bigcup_n\pi_n^{-1}(S_n\setminus Z_n)$, and thus $Z=\lim_nZ_n$. 
Each $\pi_n^{-1}(S_n\setminus Z_n)\subset S$ is {\em clopen} (closed
and open), and
$$
  U = \coprod_n\Bigl(\pi_{n+1}^{-1}(S_{n+1}\setminus Z_{n+1})\setminus
  \pi_n^{-1}(S_n\setminus Z_n)\Bigr).
$$
\end{proof}

\begin{rem}
There exist open subsets $U\subset S$ of profinite sets $S$ with
sheaf cohomology $H^i(U;\Z)\neq 0$ for some $i>0$, so $U$ is not 
a countable union of profinite sets.
\end{rem}

\begin{prop}
Let $S$ be a nonempty {\em light} profinite set. Then $S$ is an
injective object in {\bf ProFin}, i.e., for each injection $Z\subset
X$ of profinite sets each morphism $Z\to S$ extends to $X$: 
$$
\xymatrix{
  X \ar@{-->}[rd] \\ 
  Z \ar@{^{(}->}[u] \ar[r] & S
}
$$
\end{prop}

\begin{proof}
In the special case $S=\F_2$ the restriction map
$\Cont(X,\F_2)\to\Cont(Z,\F_2)$ is surjective because each clopen
subset of $Z$ extends to a clopen subset of $X$. 
In the general case, write $S=\lim_n S_n$. We may assume that each map
$S_{n+1}\to S_n$ is surjective. By induction over $n$, suppose we
already have extended the map $Z\to S_n$ to $X\to S_n$. By
decomposition into fibers over $S_n$, it suffices to consider the case
$S_n=\{*\}$. Then $S_{n+1}$ is finite and the extension follows
from the special case. 
\end{proof}

%%%
\subsection{Light condensed sets}
%%%

\begin{defi}
A {\em light condensed set} is a sheaf on the category of light
profinite sets with the Grothendieck topology generated by
\begin{enumerate}
\item finite disjoint unions;
\item {\em all} surjective maps.
\end{enumerate}
Equivalently, this is a functor 
$$
  X: \text{Pro}_\N\text{Fin}^{\rm op} \to \text{Sets},\qquad S\mapsto X(S)
$$
satisfying
\begin{enumerate}
\setcounter{enumi}{-1}
\item $X(\varnothing)=*$;
\item $X(S_1\amalg S_2)\stackrel{\sim}\longrightarrow X(S_1)\times X(S_2)$;
\item Each surjection $f:T\twoheadrightarrow S$ induces an isomorphism
$$
  f^*:X(S) \stackrel{\cong}\longrightarrow {\rm eq}\Bigl(X(T)
  \overset{p_1^*}{\underset{p_2^*}{\rightrightarrows}} X(T\times_ST)\Bigr).
$$
\end{enumerate}
\end{defi}

\begin{example}[Key example]
Each topological space $A$ defines a light condensed set 
$$
  \ul{A}: S\mapsto\Cont(S,A).
$$
Here condition (2) is equivalent to the fact that surjections between
compact Hausdorff spaces are quotient maps. We have
\begin{itemize}
\item $\ul{A}(*) = A$ as a set; 
\item $\ul{A}(\N\cup\{\infty\}) = \{\text{convergent sequences in $A$
  (with chosen limit point)}\}$;
\item $\ul{A}(\text{Cantor set})$ carries an action of
  $\End(\text{Cantor set})$. 
\end{itemize}
\end{example}

For any light condensed set $X$, we view $X(\N\cup\{\infty\})$ as the
``convergent sequences in $X$''. 

\begin{remark}[Useless remark]
A light condensed set $X$ is completely determined by $X(\text{Cantor
  set})$ with its action of $\End(\text{Cantor set})$.
\end{remark}

\begin{prop}
The functor
$$
  \text{Top} \to \text{CondSet}^{\rm light},\qquad A\mapsto\ul{A}
$$
has a left adjoint $X\mapsto X(*)_{\rm top}$. Here $X(*)_{\rm top}$ is
the set $X(*)$ equipped with the quotient topology from
$\coprod_{\alpha\in X(\text{Cantor set})}\text{Cantor set} \to X(*)$.
\end{prop}

\begin{remark}
$X(*)_{\rm top}$ is a {\em metrizably compactly generated topological
  space}. For any such space $A$ we have an isomorphism
  $\ul{A}(*)_{\rm top}\stackrel{\sim}\to A$. 
\end{remark}

\begin{cor}
The functor
$$
  \{\text{metrizably compactly generated topological space}s\} \into
  \text{CondSet}^{\rm light}
$$
is full and faithful. 
\end{cor}

\begin{rem}[Historical precursors]
1) Johnstone's ``topological topos'', based on just $\N\cup\{\infty\}$
and the canonical topology. This is not finitary and shares the bad
algebraic properties of topoi. 

2) Escordo and Xu considered light profinite sets with only finite
disjoint unions as covers. 
\end{rem}

%%%
\subsection{Light condensed abelian groups}
%%%

Recall that for sheaves on any site, sheaves of abelian groups form a
{\em Grothendieck abelian category}: all limits and colimits exist,
and filtered colimits are exact. 
Thus, $\text{CondAb}^{\rm light}$ is a Grothendieck abelian category. 

\begin{example}
What are the cokernels of the inclusions
$$
  \Q_\disc\into\R \quad\text{and}\quad \R_\disc\to \R?
$$
In the first case we have $\ul{\R}/\ul{\Q_\disc}(*)=\R/\Q$ and 
$$
  (\ul{\R}/\ul{\Q_\disc})(S) = \Cont(S,\R)/\Cont(S,\Q_\disc),
$$
where $\Cont(S,\Q_\disc)$ consists of locally constant maps.
In the second case we have $\ul{\R}/\ul{\R_\disc}(*)=\R/\R=0$ and 
$$
  (\ul{\R}/\ul{\R_\disc})(S) = \Cont(S,\R)/\Cont(S,\R_\disc),
$$
where $\Cont(S,\R_\disc)$ again consists of locally constant maps.
The abelian group $(\ul{\R}/\ul{\R_\disc})(S)$ is e.g.~nontrivial for
$S$ the Cantor set.
\end{example}

\begin{thm}
$\text{CondAb}^{\rm light}$ is a Grothendieck abelian category with
  tensor products and the following properties:
\begin{enumerate}
\item Countable products are exact and satisfy Grothendieck's axiom
  (A86).
\item The free light condensed abelian group $\Z[\N\cup\{\infty\}]$ is
  {\em internally projective}. 
\end{enumerate}
\end{thm}

Here the functor 
$$
  \text{CondAb}^{\rm light} \to \text{CondSet}^{\rm light}
$$
has a left adjoint $X\mapsto \Z[X]$. 
  
\bibliographystyle{abbrv}
\bibliography{000_condensed}

\end{document} 


%%%%%%%%%%%%%%%%%%%%%%%%%%%%%%%%%%%%%%%%%%%%%%%%%%%%%%%%%%%%%%%%%%%%%%%%%%%%%%%%%%%%%%%
%%%%%%%%%%%%%%%%%%%%%%%%%%%%%%%%%%%%%%%%%%%%%%%%%%%%%%%%%%%%%%%%%%%%%%%%%%%%%%%%%%%%%%%
%%%%%%%%%%%%%%%%%%%%%%%%%%%%%%%%%%%%%% JUNK %%%%%%%%%%%%%%%%%%%%%%%%%%%%%%%%%%%%%%%%%%%
%%%%%%%%%%%%%%%%%%%%%%%%%%%%%%%%%%%%%%%%%%%%%%%%%%%%%%%%%%%%%%%%%%%%%%%%%%%%%%%%%%%%%%%
%%%%%%%%%%%%%%%%%%%%%%%%%%%%%%%%%%%%%%%%%%%%%%%%%%%%%%%%%%%%%%%%%%%%%%%%%%%%%%%%%%%%%%%




