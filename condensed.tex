\documentclass{notes}
% LITERATURE
\addbibresource{sources.bib}
% COMMANDS
\usepackage{commands}
% THEOREMS
\usepackage{theorems}
% COLOR CODE
\usepackage{ourcolors}

\begin{document}

\title{Notes on Condensed Mathematics} 
\author{Kai Cieliebak, Milan Zerbin, ...}
\date{\today}

%%%%%%%%%%%%%%%%%%%%%%%%%%%%%%%%%%%%%%%%%%%%%%
%%%%%%%%%%%%%%%% Abstract %%%%%%%%%%%%%%%%%%%%
%%%%%%%%%%%%%%%%%%%%%%%%%%%%%%%%%%%%%%%%%%%%%%

%\begin{abstract} 
%\end{abstract}

\maketitle

%\tableofcontents

%%%%%%%%%%%%%%%%%%%%%%%%%%%%%%%%%%%%%%%%%%%%%%%%%
\section{Introduction}\label{sec:intro}
%%%%%%%%%%%%%%%%%%%%%%%%%%%%%%%%%%%%%%%%%%%%%%%%%

%%%%%%%%%%%%%%%%%%%%%%%%%%%%%%%%%%%%%%%%%%%%%%%%%
\section{Light condensed sets}\label{sec:light-condensed-sets}
%%%%%%%%%%%%%%%%%%%%%%%%%%%%%%%%%%%%%%%%%%%%%%%%%

The ordinary text in this section is a transcript of Scholze's Lecture
2 in~\cite{videos}, whose video is available under
\url{https://www.youtube.com/playlist?list=PLx5f8IelFRgGmu6gmL-Kf$\_$Rl$\_$6Mm7juZO}

Text in colours such as {\kai this one} contains details and
explanations from our side.
Current collaboratours with defined colors:
{\kai Kai}, % Red
{\milan Milan}, % Fuchsia
{\filip Filip}, % Blue
{\sadrian Adrian A}, % Dark Cyan
{\adrian Adrian W}, % Green
{\lukas Lukas},
{\luca Luca}, % Purple
\dots

Other relevant references are the lecture
notes~\cite{analytic,condensed,complex}. 

%%%
\subsection{Light profinite sets}
%%%


\textcolor{black}{}

{\adrian
\paragraph{One-point compactification of $\mathbb N$}

The usual way of defining the one-point compactification of $\mathbb N$ (as a discrete space) to $\mathbb N^* = \mathbb N \cup \{\infty\}$ is by having the previously open sets of $\mathbb N$ (all of them) be open in $\mathbb N^*$, and additionally all those sets containing $\infty$, whose complement is closed and compact inside $\mathbb N$ (equivalently, finite).

To understand $\mathbb N^*$ as a pro-finite set with the corresponding topology (inverse limit-topology), we consider the projective system of finite sets
  \[
    X_n \coloneq \{ k \in \mathbb N \mid k \leq n\} \cup \{ \infty \} = \{0, 1, \dots, n, \infty\}
  \]
  with maps $p^{n}_m \colon X_{n} \longrightarrow X_{m}$ for $n \geq m$ given by
  \[
    k \longmapsto
    \left[
    \begin{array}{lll}
      k
      & ,
      & k \leq m
      \\  \infty
      & ,
      & \textup{$k > m$, esp. if $k = \infty$}
    \end{array}
    \right.
  \]
  These maps commute, in the sense that $p^m_l \circ p^n_m = p^n_l$, which is a necessity for defining the (projective/inverse) limit of this system
  \begin{equation}
    \label{def:compactified_IN}
    \mathbb N^* \coloneq \varprojlim_{n} X_n.
  \end{equation}
  We take this limit in the category of topological spaces, where the sets $X_n$ are discrete spaces.


\paragraph{The underlying set of $\mathbb N^*$:}
Let's first check that this gives the correct set $\mathbb N \cup \{\infty\}$, by constructing a bijection.
An element of $\mathbb N^*$ can be understood as an element of the product $x = (x_n) \in \prod_n X_n$ that fulfills the compatibility conditions
  \[
    p^n_m(x_n) = x_m
  \]
  for all $m \leq n$.
Since $(X_n)$ is a linear system, it suffices to check this for the maps $p^{n+1}_n$, because all the others are compositions of those maps between ``adjacent'' sets:
  \[
    p^n_m = p^{m+1}_m \circ p^{m+2}_{m+1} \circ \cdots \circ p^n_{n-1}.
  \]
  Let's say that we have $x_n = k \neq \infty$ for some $n$.
Then on the one hand we can conclude for all smaller indices $m \leq n$ that
  \[
    x_m = p^n_m(x_n) = p^n_m(k) =
    \left[
    \begin{array}{lll}
      k
      & ,
      & k \leq m
      \\  \infty
      & ,
      & \textup{$k > m$, esp. if  $k = \infty$}
    \end{array}
    \right.,
  \]
  For all bigger indices $m \geq n$ we have $x_m = k$, since $k \neq \infty$ and therefore its only preimage under $p^m_n$ is $k \in X_m$, thus
  \[
    x = (\infty, \dots, \infty, k, k, k, \dots),
  \]
  where $x_k$ is the first finite entry.
This will be the image of $k \in \mathbb N$ under $\mathbb N \cup \{\infty\} \longrightarrow \mathbb N^*$.

The only other $x \in \mathbb N^*$ will be the one that has no finite entry, so
  \[
    x = (\infty, \infty, \dots),
  \]
  which will be the image of $\infty$, which completes the bijection.

Another thing to mention is that there are natural projections
  \[
    p_n \colon \mathbb N^* \longrightarrow X_n, x \longmapsto x_n.
  \]


  \paragraph{Topology on $\mathbb N^*$:}

  The topology on $\mathbb N^*$, viewed as an inverse limit (in the category of topological spaces), is by definition given by the coarsest topology such that these projection maps are continuous.
  (It can also be infered by restriction of the product topology on $\prod_n X_n$, but strictly speaking, this toplogy arises from the same definition, where the product is viewed as an inverse limit.)

  Since the sets $X_n$ are discrete and finite spaces, every open subsets of $\mathbb N^*$ is then a union of the preimages of single points:
  \[
      {p_n}^{-1}(k) = \{ x \mid x_n = k \} =
    \left[
    \begin{array}{lll}
      \{ k \}
      & ,
      & k \leq \infty
      \\  \{ k, k+1, \dots, \infty\}
      & ,
      & k = \infty
    \end{array}
    \right.
  \]
  Restricted to the subspace $\mathbb N$ this gives the discrete topology as expected, which means that all sets not containing $\infty$ are open, whereas sets containing $\infty$ are open if and only if they also contain every natural number bigger than some number $k \in \mathbb N$, i.e.\ their complement is closed and compact (finite) in $\mathbb N$.
This is exactly the same topology that we had put on the set $\mathbb N \cup \{\infty\}$ as the one-point compactification of $\mathbb N$, and the identification described above is therefore a homeomorphism.

}\label{subsec:light-profinite-sets}

{\lukas
We first discuss classical Stone duality between totally disconnected compact Hausdorff spaces and boolean algebras.

\begin{definition}
  A \emph{Stone space} is a totally disconnected compact Hausdorff space.
  We denote the category of Stone spaces and continuous maps by \(\Stone\).
\end{definition}

\begin{definition}
  A \emph{boolean algebra} is a commutative ring \(A\) (with unit \(1\)) such that every element \(a\in A\) is \emph{idempotent}, that is, \(a^2=a\).
  We denote the category of Boolean algebras and ring homomorphisms by \(\Bool\).
\end{definition}

\begin{example}
  The two element set \(\{0,1\}\) becomes a Stone space by equipping it with the discrete topology.
  Moreover, it carries the structure of a boolean algebra by identifying it with \(\F_2\), the finite field with two elements.
  (In fact, \(\F_2\) is a \emph{boolean algebra object} in the category of Stone spaces.)
\end{example}

\begin{example}
  For every Stone space \(X\), the set \(\Cont(X,\F_2)\) of continuous maps from \(X\) to \(\F_2\) becomes a boolean algebra with respect to pointwise addition and multiplication.
  Note that \(\Cont(X,\F_2)\) is in bijection to the set of clopen subsets of \(X\).
  (This is an isomorphism of boolean lattices?)
\end{example}

\begin{lemma}
  \leavevmode
  \begin{enumerate}
    \item
      Arbitrary products of Stone spaces are Stone spaces.
    \item
      Closed subspaces of Stone spaces are Stone spaces.
  \end{enumerate}
\end{lemma}
\begin{proof}
  By Tychonoff, arbitrary products of compact spaces are compact.
  The same holds for Hausdorff spaces and totally disconnected spaces (?).
  \dots
\end{proof}

\begin{example}
  Let \(A\) be a boolean algebra.
  The set \(\Hom(A,\F_2)\) of ring homomorphisms from \(A\) to \(\F_2\) is a closed subset of \(\prod_{a\in A}\F_2\) equipped with the product topology.
  Endowing it with the induced subspace topology, it becomes a Stone space which is called the \emph{spectrum} of \(A\) and denoted by \(\Spec(A)\).

  Missing: Show that this is homeomorphic to the Zariski spectrum!
  Under this homeomorphism, the standard open subsets \(D(a)\), where \(a\in A\), correspond to open sets of the form
  \[\{\phi\in\Spec(A) \mid \phi(a)=1\},\]
  which are henceforth denoted by \(D(a)\) as well.
  In fact, \(D(a)\) is also a closed set, since \(\Spec A\setminus D(a)=D(1-a)\).
  Consequently, the spectrum of a boolean algebra possesses a basis of clopen subsets, which gives another proof that it is totally disconnected.
\end{example}

\begin{thm}[Stone Duality I]
  Assigning to a Stone space \(X\) the boolean algebra \(\Cont(X,\F_2)\), and to a boolean algebra \(A\) its spectrum \(\Spec(A)\), constitutes an equivalence of categories
  \[
    \Stone \simeq \Bool^{\rm op}.
  \]
\end{thm}
\begin{proof}
  First of all, the assignments \(X\to\Cont(X,\F_2)\) and \(A\mapsto\Spec(A)=\Hom(A,\F_2)\) are obviously (contravariantly) functorial (by precomposition).
  We are then to show that they are mutually (pseudo)inverse to each other, that is, we have to find natural isomorphisms \(X\cong\Spec(\Cont(X,\F_2))\) and \(A\cong\Cont(\Spec A,\F_2)\).
  Both of these are defined as evaluation maps.

  More precisely, let us first consider the map
  \[e:X \longrightarrow \Spec(\Cont(X,\F_2)), \quad x \longmapsto (e_x:f \mapsto f(x)).\]
  Obviously \(e_x\) is a morphism of boolean algebras.
  To show that \(e\) is continuous, let \(f\in \Cont(X,\F_2)\) and \(D(f)\) the associated standard clopen set in \(\Spec(A)\).
  Its preimage under \(e\) is the set of all \(x\in X\) such that \(f(x)=e_x(f)=1\).
  This condition is equivalent to \(f(x)\neq 0\) which is an open condition, so \(e\) is continuous.
  Missing: Bijectivity, naturality.

  Next, consider the evaluation map
  \[e:A \longrightarrow  \Cont(\Spec A,\F_2),\quad a\mapsto (e_a : \phi\mapsto \phi(a))\]
  which is obviously a ring homomorphism.
  To prove injectivity, suppose that \(a\neq 0\).
  By Zorn's lemma, there is a maximal ideal \(I\) of \(A\) which contains \(1-a\neq 1\).
  Since quotient of boolean rings are boolean, and quotients of rings by maximal ideals are fields, the quotient ring \(A/I\) is (uniquely) isomorphic to \(\F_2\).
  By composing this isomorphism with the canonical projection \(A\to A/I\) we obtain a morphism of boolean algebras \(\phi:A\to \F_2\) with kernel \(I\).
  In particular, \(\phi(1-a)=0\), or equivalently, \(\phi(a)=1\), so \(e_a\neq 0\) which proves injectivity.

  For surjectivity, let \(f:\Spec A \to \F_2\) be any continuous function.
  Such a function is uniquely determined by the clopen set \(U=f^{-1}(1)\subseteq\Spec(A)\).
  If we can find an \(a\in A\) with \(U=D(a)\), then \(f\) is exactly \(e_a\).
  Since \(U\) is open, it can be written as a union of standard open sets.
  But since \(U\) is closed and thus compact, finitely many of these suffice, say, \(U=D(a_1)\cup\dots\cup D(a_n)\).
  Recall from above that in a boolean algebra, \(\Spec(A)\setminus D(a)=D(1-a)\).
  Moreover, the standard opens satisfy the property \(D(a)\cap D(b)=D(ab)\) (this holds in the spectrum of every ring).
  Therefore, if we define \(a\) as \(1-(1-a_1)\cdots (1-a_n)\), we obtain
  \[
    D(a) = \Spec(A)\setminus (D(1-a_1)\cap\dots\cap D(1-a_n)) = D(a_1)\cup\dots\cup D(a_n) = U.
  \]
  [The proof of surjectivity corresponds to Exercise 23 (iii) in Atiyah-MacDonald.]
\end{proof}
}

{\luca
\paragraph{Grothendieck Topologies}

\def\C{\mathbf{C}}
\def\Set{\mathbf{Set}}

\def\op{\textup{op}}

Let $\C$ be a (small) category.

\begin{definition}
	A \emph{sieve} S on $U \in \C$ is a family of morphisms with codomain $U$, such that
	\begin{equation*}
		f \in S \quad \Longrightarrow \quad f \circ g \in S
	\end{equation*}
	for all $g$ composable with $f$.
	
	Equivalently a sieve can be described as a subobject $S \subset h_U \coloneq \Hom_\C(\cdot,U)$ in the category of presheafs $\widehat\C = \{\text{functors : } \C^\op \rightarrow \Set\}$.
	
	We denote the set of all sieves on $U$ by $\Omega(U)$.
\end{definition}

For a sieve $S$ on $U$ and a morphism $f : V \rightarrow U$ the restriction along $f$
\begin{equation*}
	f^*S \coloneq \{g \in h_V \; | \; f \circ g \in S\}
\end{equation*}
clearly is a sieve on $V$.
Using this one can check that $\Omega$ defines a presheaf by setting $\Omega(f) = f^*$.

\begin{definition}
	A \emph{Grothendieck topology} $J$ on $\C$ assigns to each object $U$ a set $J(U) \subset \Omega(U)$ of so called \emph{covering sieves}, such that
	\begin{enumerate}
		\item the maximal sieves $h_U$ is in $J(U)$;
		\item (stability) if $S \in J(U)$, then $f^*S \in J(V)$ for all $f : V \rightarrow U$;
		\item (transitivity) if $S \in J(U)$ and $R \in \Omega(U)$ such that $f^*R \in J(V)$ for all $f : V \rightarrow U$ in $S$, then $R \in J(U)$.
	\end{enumerate}
	The pair $(\C,J)$ is then called a \emph{site}.
\end{definition}

Because of stability, $J$ is a subobject of $\Omega$ in $\widehat{\C}$.

Here we want to define a Grothendieck topology generated by ``covering families'' that are no sieves in general. For this we assume $\C$ to have pullbacks.

\begin{definition}
	A \emph{basis} $K$ on $\C$ assigns to each object $U$ a set $K(U)$ of families of morphisms with codomain $U$, so called \emph{covers}, such that
	\begin{enumerate}
		\item every isomorphism $\{f : V \rightarrow U\}$ is in $K(U)$;
		\item (stability) if $\{f_i : U_i \rightarrow U\} \in K(U)$, then for all $g : V \rightarrow U$ the family of pullbacks along $g$ $\{g^*f_i : U_i \times_U V \rightarrow U\}$
		is in $K(V)$.
		\item (transitivity) if $\{f_i : U_i \rightarrow U\} \in K(U)$ and for each $i$ one has a cover $\{f_{ij} : U_{ij} \rightarrow U_i\} \in K(U_i)$, then the ``refinement'' $\{f_i \circ f_{ij} : U_{ij} \rightarrow U\}$ is in $K(U)$.
	\end{enumerate}
\end{definition}
}

Consider the following three categories.

1) {\em Profinite sets} {\bf ProFin} \\
Objects: $\lim_{i\in I}S_i$ for $S_i$ finite sets, $I$ cofiltered poset
($\forall i,j \exists k\leq i,j$)
\\
Morphisms: 
$$
  \Hom(\lim_{i\in I}S_i,\lim_{j\in J}T_j) := \lim_j\colim_i\Map(S_i,T_j)
$$

2) {\em Totally disconnected compact Hausdorff spaces} {\bf DiscComp}

3) {\em Boolean algebras} {\bf Boolean} \\ 
Objects: commutative rings $A$ such that $x^2=x$ for all $x$ 

\begin{prop}[Stone Duality]
There are equivalences of categories
$$
  \textbf{ProFin} \stackrel{\sim}\longrightarrow 
  \textbf{DiscComp} \stackrel{\sim}\longrightarrow  
  \textbf{Boolean}^{\rm op}.
$$
\end{prop}

\begin{proof}
For 1) $\to$ 2) $\to$ 3), equip $S=\lim S_i$ with the (inverse,
projective) limit topology and assign to it the Boolean algebra
$$
   \Cont(S,\F_2) = \colim_i\Map(S_i,\F_2).
$$
Conversely, assign to $A$ the profinite set
$$
  \Spec A = \Hom(A,\F_2) = \lim_{A_i\subset A\text{ finite}}\Hom(A_i,\F_2).
$$
\end{proof}

There are two measures how ``big'' a profinite set is.

\begin{defi}
Let $S$ be a profinite set. 
\begin{itemize}
\item The {\em size} of $S$ is $\kappa=|S|$.
\item The {\em weight} of $S$ is $\lambda=|\Cont(S,\F_2)|$.
\item $S$ is {\em light} if $\lambda\leq\om=|\N|$.
\end{itemize}
\end{defi}

\begin{rem}
If $\lambda$ is infinite, then $\lambda$ is the smallest
  cardinality of an index set $I$ such that $S=\lim_{i\in I}S_i$.
Thus, $S$ is light iff is is a countable limit of finite sets.
\end{rem}

\begin{example}[Examples of profinite sets]
(0) Finite sets.
  
(1) $S = \N\cup\{\infty\} = \lim_n\{0,1,\dots,n,\infty\}$ has $\kappa=\lambda=\om$. 

(2) The {\em Cantor set} $S = \{0,1\}^\N = \lim_n\{0,1\}^n$ has
  $\kappa=2^\om$ and $\lambda=\om$.

(3) The {\em Stone-\v{C}ech compactification} $S=\beta\N$ satisfies 
$$
  \Cont(\beta\N,\F_2) = \Cont(\N,\F_2) \cong \{\text{subset of }\N\},
$$
so it has $\kappa=2^{2^\om}$ and $\lambda=2^\om$. 
\end{example}

\begin{prop}
(a) We always have $\lambda\leq 2^\kappa$ and $\kappa\leq 2^\lambda$. \\
(b) If $\kappa=\om$ then $\lambda=\om$. 
\end{prop}

\begin{proof}
(a) follows from
\begin{align*}
  \lambda &= |\Cont(S,\F_2)| \leq |\Map(S,\F_2)| = 2^\kappa, \cr
  \kappa &= |\Hom(A,\F_2)| \leq |\Map(A,\F_2)| = 2^\lambda.
\end{align*}
(b) Write $S=\{s_0,s_1,s_2,\dots\}$ and choose quotients
$S\twoheadrightarrow S_n$ which restrict to injections
$\{s_0,\dots,s_n\}\into S_n$. Then the induced map $S\to\lim_n S_n$ is
injective and surjective.
%and thus $\Cont(S,\F_2)=\colim_n\Cont(S_n,\F_2)$ is countable.   
\end{proof}

\begin{rem}
The same proof shows: if $\kappa$ is infinite then $\lambda\leq\kappa$.
\end{rem}

\begin{prop}
The following categories are equivalent to the category of light
profinite sets:

1) $\text{\bf Pro}_{\N}\text{\bf Fin}$ \\
Objects: $\lim_{n\in\N}S_n$ for $S_n$ finite sets
\\
Morphisms: 
\begin{align*}
  \Hom(\lim_{n\in\N}S_n,\lim_{m\in\N}T_m)
  &:= \lim_m\colim_n\Map(S_n,T_m) \cr
  &= \colim_f\lim_m\Map(S_{f(m)},T_m)
\end{align*}
over strictly increasing maps $f:\N\to\N$.
  
2) {\em Metrizable} totally disconnected compact Hausdorff spaces

3) $(\text{{\em Countable} Boolean algebras})^{\rm op}$
\end{prop}

\begin{prop}
The category of light profinite sets has all countable
limits. Sequential limits of surjections are surjective. 
\end{prop}

\begin{prop}
For each light profinite set $S$ there exists a surjection
$\{0,1\}^\N\to S$. 
\end{prop}

\begin{proof}
Exercise (simple induction).
\end{proof}

The following two propositions are special to {\em light} profinite
sets. 

\begin{prop}
Let $S$ be a {\em light} profinite set. Then every open subset $U\subset S$
is a countable union of light profinite sets. 
\end{prop}

\begin{proof}
Write $S=\lim_{n\in\N}S_n$ with $S_n$ finite and the canonical maps
$\pi_n:S\to S_n$. Denote $Z:=S\setminus U$ and $Z_n:=\pi_n(Z)\subset S_n$.
Then $U=\bigcup_n\pi_n^{-1}(S_n\setminus Z_n)$, and thus $Z=\lim_nZ_n$. 
Each $\pi_n^{-1}(S_n\setminus Z_n)\subset S$ is {\em clopen} (closed
and open), and
$$
  U = \coprod_n\Bigl(\pi_{n+1}^{-1}(S_{n+1}\setminus Z_{n+1})\setminus
  \pi_n^{-1}(S_n\setminus Z_n)\Bigr).
$$
\end{proof}

\begin{rem}
There exist open subsets $U\subset S$ of profinite sets $S$ with
sheaf cohomology $H^i(U;\Z)\neq 0$ for some $i>0$, so $U$ is not 
a countable union of profinite sets.
\end{rem}

\begin{prop}
Let $S$ be a nonempty {\em light} profinite set. Then $S$ is an
injective object in {\bf ProFin}, i.e., for each injection $Z\subset
X$ of profinite sets each morphism $Z\to S$ extends to $X$: 
$$
\xymatrix{
  X \ar@{-->}[rd] \\ 
  Z \ar@{^{(}->}[u] \ar[r] & S
}
$$
\end{prop}

\begin{proof}
In the special case $S=\F_2$ the restriction map
$\Cont(X,\F_2)\to\Cont(Z,\F_2)$ is surjective because each clopen
subset of $Z$ extends to a clopen subset of $X$. 
In the general case, write $S=\lim_n S_n$. We may assume that each map
$S_{n+1}\to S_n$ is surjective. By induction over $n$, suppose we
already have extended the map $Z\to S_n$ to $X\to S_n$. By
decomposition into fibers over $S_n$, it suffices to consider the case
$S_n=\{*\}$. Then $S_{n+1}$ is finite and the extension follows
from the special case. 
\end{proof}

%%%
\subsection{Light condensed sets}
%%%

\begin{defi}
A {\em light condensed set} is a sheaf on the category of light
profinite sets with the Grothendieck topology generated by
\begin{enumerate}
\item finite disjoint unions;
\item {\em all} surjective maps.
\end{enumerate}
Equivalently, this is a functor 
$$
  X: \text{Pro}_\N\text{Fin}^{\rm op} \to \text{Sets},\qquad S\mapsto X(S)
$$
satisfying
\begin{enumerate}
\setcounter{enumi}{-1}
\item $X(\varnothing)=*$;
\item $X(S_1\amalg S_2)\stackrel{\sim}\longrightarrow X(S_1)\times X(S_2)$;
\item Each surjection $f:T\twoheadrightarrow S$ induces an isomorphism
$$
  f^*:X(S) \stackrel{\cong}\longrightarrow {\rm eq}\Bigl(X(T)
  \overset{p_1^*}{\underset{p_2^*}{\rightrightarrows}} X(T\times_ST)\Bigr).
$$
\end{enumerate}
\end{defi}

\begin{example}[Key example]
Each topological space $A$ defines a light condensed set 
$$
  \ul{A}: S\mapsto\Cont(S,A).
$$
Here condition (2) is equivalent to the fact that surjections between
compact Hausdorff spaces are quotient maps. We have
\begin{itemize}
\item $\ul{A}(*) = A$ as a set; 
\item $\ul{A}(\N\cup\{\infty\}) = \{\text{convergent sequences in $A$
  (with chosen limit point)}\}$;
\item $\ul{A}(\text{Cantor set})$ carries an action of
  $\End(\text{Cantor set})$. 
\end{itemize}
\end{example}

For any light condensed set $X$, we view $X(\N\cup\{\infty\})$ as the
``convergent sequences in $X$''. 

\begin{remark}[Useless remark]
A light condensed set $X$ is completely determined by $X(\text{Cantor
  set})$ with its action of $\End(\text{Cantor set})$.
\end{remark}

\begin{prop}
The functor
$$
  \text{Top} \to \text{CondSet}^{\rm light},\qquad A\mapsto\ul{A}
$$
has a left adjoint $X\mapsto X(*)_{\rm top}$. Here $X(*)_{\rm top}$ is
the set $X(*)$ equipped with the quotient topology from
$\coprod_{\alpha\in X(\text{Cantor set})}\text{Cantor set} \to X(*)$.
\end{prop}

\begin{remark}
$X(*)_{\rm top}$ is a {\em metrizably compactly generated topological
  space}. For any such space $A$ we have an isomorphism
  $\ul{A}(*)_{\rm top}\stackrel{\sim}\to A$. 
\end{remark}

\begin{cor}
The functor
$$
  \{\text{metrizably compactly generated topological space}s\} \into
  \text{CondSet}^{\rm light}
$$
is full and faithful. 
\end{cor}

\begin{rem}[Historical precursors]
1) Johnstone's ``topological topos'', based on just $\N\cup\{\infty\}$
and the canonical topology. This is not finitary and shares the bad
algebraic properties of topoi. 

2) Escordo and Xu considered light profinite sets with only finite
disjoint unions as covers. 
\end{rem}

%%%
\subsection{Light condensed abelian groups}
%%%

Recall that for sheaves on any site, sheaves of abelian groups form a
{\em Grothendieck abelian category}: all limits and colimits exist,
and filtered colimits are exact. 
Thus, $\text{CondAb}^{\rm light}$ is a Grothendieck abelian category. 

\begin{example}
What are the cokernels of the inclusions
$$
  \Q_\disc\into\R \quad\text{and}\quad \R_\disc\to \R?
$$
In the first case we have $\ul{\R}/\ul{\Q_\disc}(*)=\R/\Q$ and 
$$
  (\ul{\R}/\ul{\Q_\disc})(S) = \Cont(S,\R)/\Cont(S,\Q_\disc),
$$
where $\Cont(S,\Q_\disc)$ consists of locally constant maps.
In the second case we have $\ul{\R}/\ul{\R_\disc}(*)=\R/\R=0$ and 
$$
  (\ul{\R}/\ul{\R_\disc})(S) = \Cont(S,\R)/\Cont(S,\R_\disc),
$$
where $\Cont(S,\R_\disc)$ again consists of locally constant maps.
The abelian group $(\ul{\R}/\ul{\R_\disc})(S)$ is e.g.~nontrivial for
$S$ the Cantor set.
\end{example}

\begin{thm}
$\text{CondAb}^{\rm light}$ is a Grothendieck abelian category with
  tensor products and the following properties:
\begin{enumerate}
\item Countable products are exact and satisfy Grothendieck's axiom
  (A86).
\item The free light condensed abelian group $\Z[\N\cup\{\infty\}]$ is
  {\em internally projective}. 
\end{enumerate}
\end{thm}

Here the functor 
$$
  \text{CondAb}^{\rm light} \to \text{CondSet}^{\rm light}
$$
has a left adjoint $X\mapsto \Z[X]$. 
  
\printbibliography

\end{document} 


%%%%%%%%%%%%%%%%%%%%%%%%%%%%%%%%%%%%%%%%%%%%%%%%%%%%%%%%%%%%%%%%%%%%%%%%%%%%%%%%%%%%%%%
%%%%%%%%%%%%%%%%%%%%%%%%%%%%%%%%%%%%%%%%%%%%%%%%%%%%%%%%%%%%%%%%%%%%%%%%%%%%%%%%%%%%%%%
%%%%%%%%%%%%%%%%%%%%%%%%%%%%%%%%%%%%%% JUNK %%%%%%%%%%%%%%%%%%%%%%%%%%%%%%%%%%%%%%%%%%%
%%%%%%%%%%%%%%%%%%%%%%%%%%%%%%%%%%%%%%%%%%%%%%%%%%%%%%%%%%%%%%%%%%%%%%%%%%%%%%%%%%%%%%%
%%%%%%%%%%%%%%%%%%%%%%%%%%%%%%%%%%%%%%%%%%%%%%%%%%%%%%%%%%%%%%%%%%%%%%%%%%%%%%%%%%%%%%%




